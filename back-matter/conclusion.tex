\clearpage{\pagestyle{empty}\cleardoublepage}
\chapter*{Conclusioni}
\rhead[\fancyplain{}{\bfseries
CONCLUSIONI}]{\fancyplain{}{\bfseries\thepage}}
\lhead[\fancyplain{}{\bfseries\thepage}]{\fancyplain{}{\bfseries
CONCLUSIONI}}
\addcontentsline{toc}{chapter}{Conclusioni}
\noindent 
Questa tesi è nata con lo scopo di creare un'infografica responsive dedicata a promuovere processi di dematerializzazione per l'università di Bologna.\newline Tutti gli obiettivi prefissati sono stati raggiunti e le tempistiche sono state rispettate. Lo scopo principale del progetto è stato raggiunto, tuttavia sono numerose le migliorie che si potrebbero apportare.\newline
L'implementazione delle Homepage per i Campus UniBo rimanenti consentirebbero la visione dei dati ad una cerchia molto più ampia di persone.\newline
La suddivisione delle pagine in componenti renderà sicuramente più semplice l'aggiunta futura di nuovi dati e di nuovi progetti.
Non è da escludere l'idea della creazione di un'applicazione dedicata, oppure di un'integrazione con App UniBo già esistenti.\newline
Feature come notifiche o tecniche di gamification potrebbero essere ottime aggiunte per un semplice applicativo che si occupa semplicemente di presentare dati come questo.\newline
Molti insegnamenti offerti dall'Università di Bologna contribuiscono al perseguimento degli Obiettivi di Sviluppo Sostenibile (SDGs) dell'Agenda 2030 dell'ONU.
Per esempio, il corso di Ingegneria e Scienze Informatiche triennale ``Programmazione di Sistemi Mobile"  contribuisce al raggiungimento degli SDG 4, 8, 9 e 11.
Sebbene non sia lo scopo principale dell'infografica, si potrebbe considerare l'aggiunta di una pagina che mostra tutti i corsi di studio che puntano a raggiungere determinati SDG.\newline
Le condizioni atmosferiche offerte dall'API di OpenWeatherMap sono molte più di quelle create fin'ora. L'implementazione delle condizioni atmosferiche rimanenti (come il temporale) aumenterebbero il livello di dettaglio dell'interfaccia.\newline
Un altro dettaglio che potrebbe essere aggiunto sono le fasi lunari nel tema notturno, calcolabili anche senza l'ausilio di un'API aggiuntiva.
Si potrebbe considerare l'aggiunta di una sezione dedicata alle notizie riguardanti aspetti di sostenibilità dell'Università di Bologna o generali.\newline
Gli eventuali sviluppi futuri di questa infografica sono numerosi, ma è importante ricordare che mantenere la semplicità è fondamentale per garantire una buona user experience.
