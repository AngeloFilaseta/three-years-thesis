\pagenumbering{roman}
\chapter*{Introduzione}
\rhead[\fancyplain{}{\bfseries
INTRODUZIONE}]{\fancyplain{}{\bfseries\thepage}}
\lhead[\fancyplain{}{\bfseries\thepage}]{\fancyplain{}{\bfseries
INTRODUZIONE}}
\addcontentsline{toc}{chapter}{Introduzione}
\noindent L'organizzazione delle Nazioni Unite ha sottoscritto nel 2015 l'Agenda 2030 per lo sviluppo sostenibile. Si tratta di un piano d'azione per garantire un presente e un futuro migliore al nostro Pianeta e alle persone che lo abitano.\newline
Lo sviluppo sostenibile mira a soddisfare i
bisogni della generazione presente senza compromettere la capacità delle future di soddisfare i propri.\newline
Per raggiungere uno sviluppo sostenibile è però importante armonizzare tre
elementi fondamentali: la crescita economica, l’inclusione sociale e la tutela dell’ambiente.
In questa tesi verrà trattato principalmente questo ultimo punto.
La deforestazione e la desertificazione, causate principalmente dalle attività dell’uomo e dal cambiamento climatico, pongono sfide considerevoli in termini di sviluppo sostenibile, e hanno condizionato le vite e i mezzi di sostentamento di milioni di persone che lottano contro la fame e la povertà.

\noindent Dal 2016 anche l’Università di Bologna ha potenziato la digitalizzazione dei processi amministrativi e di comunicazione per incrementare il risparmio della carta e favorire il benessere ambientale.\newline
In particolare, il campus di Cesena si sta impegnando per creare un'area dedicata alla piantumazione di nuovi alberi. Il piccolo bosco si troverà nei pressi dell'ingresso sul retro dello stabile.\newline
L'obiettivo finale di questa tesi è la creazione di un'infografica responsive dedicata a promuovere tutti questi processi di dematerializzazione.\newline
Scopo dell'infografica è mostrare dati, in particolare il numero di fogli di carta, di alberi, e il quantitativo di CO$_2$ che un determinato processo è riuscito a risparmiare negli anni.

\noindent Partendo da alcuni mockup è stata sviluppata un'interfaccia adatta ad ogni tipo di schermo.\newline
L'applicazione ha dovuto soddisfare un certo grado di scalabilità.
L'aggiunta di nuovi progetti sostenibili all'infografica non dovrà essere, in futuro, un problema da gestire modificando l'intera struttura delle pagine.
Anche l'effettiva l'implementazione di feature supplementari non dovrà essere dipendente dalla struttura già esistente.\newline 
Il modo in cui vengono presentati i dati e le modalità d'uso sono semplici e intuitive. Qualsiasi tipo di utente deve infatti poter essere in grado di utilizzare l'infografica senza averla mai vista prima.\newline

\noindent Il seguito della tesi è così organizzato:
\begin{itemize}
    \item Nel primo capitolo viene introdotto il tema della sostenibilità. Vengono illustrati gli eventi istituiti dall'ente delle Nazioni Unite relativi al tema che sono stati di maggior rilevanza nella storia. Vengono descritti gli obiettivi di sviluppo sostenibile con un occhio di riguardo per il problema del riscaldamento globale.
    Vengono inoltre trattati i temi Data Visualization e  Interactive Public Ambient Display. Il primo tratta di metodologie per mostrare determinati tipi di dati ad un utente visitatore, il secondo è un progetto di interazione tra persone e Display pubblici.
    \item Nel secondo capitolo vengono illustrati tutti gli strumenti che sono stati utilizzati per creare l'infografica.
    Vengono innanzitutto illustrati gli strumenti da cui più dipende il lato grafico, passando poi al vero e proprio Framework su cui è basato tutto il progetto, React.js. Infine vengono descritte alcune librerie utilizzate in aggiunta.
    \item Nel terzo capitolo viene descritto il vero e proprio sviluppo dell'infografica. Dopo aver fatto alcune considerazioni durante la fase di analisi si passa alla strutturazione del progetto e delle pagine che lo compongono. Successivamente vegonono aggiunti i componenti logici e di personalizzazione dell'interfaccia.
\end{itemize}

\clearpage{\pagestyle{empty}\cleardoublepage}